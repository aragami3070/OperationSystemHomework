\documentclass[bachelor, och, referat, times]{SCWorks}

\usepackage[T2A]{fontenc}
\usepackage[utf8]{inputenc}
\usepackage{graphicx}
\usepackage[sort,compress]{cite}
\usepackage{amsmath}
\usepackage{amssymb}
\usepackage{amsthm}
\usepackage{fancyvrb}
\usepackage{longtable}
\usepackage{array}
\usepackage{makecell}
\usepackage{multirow}
\usepackage[english,russian]{babel}

\usepackage{tempora}
\usepackage[hidelinks]{hyperref}

\usepackage{pgfplots}
\usepackage{tikz}
\usepackage{float}
\pgfplotsset{compat = newest}

\usepackage{minted}
\setminted[c++]{linenos, breaklines = true, style = bw, fontsize = \small}

\usepackage{forloop}
\usepackage{pgffor}

\begin{document}

    \chair{информатики и программирования}
    % Тема работы
    \title{Домашнее задание 0 по Операционным системам}
    
    % Курс
    \course{2}
    
    % Группа
    \group{251}
    
    % Факультет (в родительном падеже) (по умолчанию "факультета КНиИТ")
    \department{факультета компьютерных наук и информационных технологий}
    
    % Специальность/направление код - наименование
    
    \napravlenie{09.03.04 "--- Программная инженерия}
    
    % Для студентки. Для работы студента следующая команда не нужна.
    \studenttitle{студента}
    
    % Фамилия, имя, отчество в родительном падеже
    \author{Смирнова Егора Ильича}
    
    \date{2024}
    
    \maketitle
% \vspace*{8cm}
%
% \begin{center}
%     % \scalebox{3}{\textbf{ПИздец}} 
%
%     \vspace{0.5cm}
%    
%     \scalebox{2}{\textbf{Домашнее задание 1 по БЖД 3 семестр}} 
% \end{center}
%


\tableofcontents

% Оптимизация от Дани
\foreach \n in {1,...,6}{
    \input{tasks/task \n}
}


\end{document}
